\documentclass[11pt]{article}
\usepackage{graphicx}
\usepackage[T1]{fontenc}
\usepackage[utf8]{inputenc}
\usepackage{geometry}
\pagestyle{empty}
\usepackage{amsmath}
\usepackage{minted}
\usepackage[albanian]{babel}
\usepackage{listings}
\usepackage{color}
\usepackage{setspace}
\onehalfspacing


\definecolor{dkgreen}{rgb}{0,0.6,0}
\definecolor{gray}{rgb}{0.5,0.5,0.5}
\definecolor{mauve}{rgb}{0.58,0,0.82}




\lstset{frame=tb,
  language=C++,
  aboveskip=3mm,
  belowskip=3mm,
  showstringspaces=false,
  columns=flexible,
  basicstyle={\small\ttfamily},
   numbers=left,   
  numberstyle=\tiny\color{gray},
  keywordstyle=\color{blue},
  commentstyle=\color{dkgreen},
  stringstyle=\color{mauve},
  breaklines=true,
  breakatwhitespace=true,
  tabsize=3
}




\usepackage{titlesec}
\usepackage{titling}


\titleformat{\section}
{\Large}
{\hspace{0.2in}}
{1em}
{}



\begin{document}
\begin{center}
\line(1,0){423}\\
[0.25in]
    \huge{\bfseries\textsc{Arkitetura e Kompjuterëve}}\\
[2mm]
\line(1,0){300}\\
[1.5cm]
\end{center}
\begin{center}
\includegraphics[scale=0.16]{up.png}
\end{center}
\vspace{2cm}
\begin{center}
\textsc{{\bfseries\LARGE Detyra 1}}\\
%[2 cm]
%\textsc{{\large Valon Raca}}\\
\vspace{2.3cm}
\end{center}

\begin{flushright}
\line(1,0){100pt}\\
[0.3cm]
\textsc{Osman Bytyqi}\\
\# 190714100081\\
April 2022\\
\line(1,0){65pt}\\
\end{flushright}
\newpage

\section{Kodi në gjuhën C++}

Kodi në vazhdim  paraqet një funksion i cili si vlere pranon parametrin  test ku është hardcoduar me vleren 4, më pas funksioni  do të shtyp  të gjithë numrat me rënditje prej 4 deri në 1 dhe  prej 1 deri ne 4.\\

Kusht i nevojshëm  që ky funksion të vazhdoj  me ekzekutim është qe vlera e variables  test apo parametrit  ne këtë rast te jetë më e madhe se 1 përndryshe programi do të ndaloj ekzekutuari, ku këtë kusht e kemi arritur përmes nje if bloku duke e kontrolluar që test variabla të jetë më e madhe se $1$, dhe në fund e kemi thirrur në funksionin main.
\vspace{2cm}
\begin{lstlisting}

#include<iostream>
using namespace std;
void printFun(int test){
    if (test<1){
        return  ;
    }
    else {
        cout<<test<<" ";
        printFun(test-1);
        cout<<test<<" ";
        return ;
    }
}
int main(){
    int test=4;
    printFun(test);
}


\end{lstlisting}
\newpage
\section{Realizimi i Kodit në Mips}
\begin{minted}{gas}
.text
.globl main

main:
addi $s0,$zero,4
add $a0,$s0,$zero      # vendosja e vleres se regjistrit s0 ne argument
jal printFun           #thirrja  e funkstionit
printFun:
addi $sp,$sp,-8        # rezervimi i hapsires ne stack 
sw $ra,4($sp)          # rezervimi i vleres return ne stack
sw $s0,0($sp)          # rezervimi i vleres se parametrit ne stack
slti $t0,$s0,1         #test for n<1
beq $t0,$zero,else     #if test>=1 go to else 
beq $t0,1 else2
addi $sp,$sp,8         #lirimi i hapsires se stackut 

printfun2:
addi $sp,$sp,-8        # rezervimi i hapsires ne stack 
sw $ra,4($sp)          # rezervimi i vleres return ne stack
sw $s0,0($sp)          # rezervimi i vleres se parametrit ne stack
slti $t0,$s0,4         #test for n<1
beq $t0,1 else2
addi $sp,$sp,8         #lirimi i hapsires se stackut 
li $v0,10
syscall

else:
li $v0,1               # print test 
move $a0,$s0
syscall

li $v0,4               #print space
la $a0,space           #load address of space
syscall

addi $s0,$s0,-1        #t-1
sw $s0,0($sp)          #ruajta ne stack 
j printFun             #kcimi ne printfun
addi $sp,$sp,8         #zbrasja e stackut

li $v0,1               # print test 
move $a0,$a0
syscall

li $v0,4              #print space
la $a0,space          #load address of space
syscall
addi $v0,$zero,0       #return 
jr $ra

else2:
addi $s0,$s0,1
li $v0,1               # print test 
move $a0,$s0
syscall

li $v0,4               #print space
la $a0,space           #load address of space
syscall

sw $s0,0($sp)          #ruajta ne stack 
j printfun2             #kcimi ne printfun
addi $sp,$sp,8         #zbrasja e stackut

.data
space: .asciiz " "


      
\end{minted}


\begin{figure}[h]
    \centering
    \includegraphics[width=0.9\textwidth]{test1.png}
%  \includegraphics[widht=0.01]{test1.png}
    \caption{Testimi}
    \label{testimi}
\end{figure}

\vspace{1cm}
Përpos implementimit të
logjikës së menyrës së zbritjes së parametrit
një pjesë tjetër interesante ka
qenë puna me stack e cila ka dashur një koncentrim më të madh.
\vspace{2cm}
\newpage
\section{Përfundimi}
Arkitektura e Kompjuterëve është ndër lëndët e vetme deri tash ku nuk kam pasur aq njohuri
paraprake, andaj edhe zgjodha opsionin e parë në mënyrë që të hulumtoj më shumë dhe të mësoj
më mirë.

Përfundimi i detyrës nuk ishte edhe aq i lehtë, mirëpo kjo ndikoi që të përvetësoj ato që janë
shpjeguar në ligjërata dhe ushtrime dhe t’i kuptoj ato që nuk i kam pasur edhe aq të qarta, çka edhe
është qëllim i detyrave të tilla.
Në ndihmë kanë ardhur artikuj të ndryshëm në Google dhe inqizime të llojllojshme në YouTube.
Me kombinim të të gjitha veglave ndihmëse, zgjidhja e detyrës mori punë disa orëshe brenda disa
ditëve



\end{document}
